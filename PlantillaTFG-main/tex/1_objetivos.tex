\capitulo{1}{Objetivos}

El principal objetivo del proyecto es construir un modelo de red neuronal convolucional capaz de predecir el grado de retinopatía diabética de un paciente a partir de una imagen de su fondo de ojo.

Se buscará llevar a cabo la construcción íntegra de la red, partiendo desde cero y sin hacer uso de \textit{transfer learning}.

\section{Objetivos de software}
\begin{itemize}[itemsep=0.25em]
    \item Desarrollar una arquitectura de red neuronal convolucional capaz de predecir el grado de retinopatía diabética a partir de una imagen de fondo de ojo.
    \item Construir la arquitectura sin emplear \textit{transfer learning}.
    \item La red neuronal deberá ser capaz de realizar diagnósticos a partir de imágenes tomadas con retinógrafos así como con dispositivos móviles (tanto Android como iOS).
\end{itemize}

\section{Objetivos técnicos}
\begin{itemize}[itemsep=0.25em]
    \item Evaluar el rendimiento de la red neuronal en el diagnóstico con imágenes de dispositivos móviles empleando las métricas \textit{Quadratic Weighted Kappa}, \textit{Balanced Accuracy}, Valor-F o \textit{F-score} y el área bajo la curva ROC.
    \item Alcanzar un valor en las métricas mencionadas superior al obtenido por el clínico sobre el mismo conjunto de imágenes de dispositivos móviles. Se tomará como \textit{gold standard} el diagnóstico clínico sobre las imágenes tomadas con retinógrafos.
\end{itemize}

\section{Objetivos de aprendizaje}
\begin{itemize}[itemsep=0.25em]
    \item Desarrollar el proyecto dentro del \textit{framework} PyTorch, empleando el lenguaje de programación Python.
    \item Emplear GitHub para crear un repositorio en el que almacenar el código y los resultados que se vayan generando a lo largo del proyecto.
    \item Conocer el modelo de de trabajo en minería de datos CRISP-DM, y seguir su metodología en el desarrollo del proyecto.
    \item Entrenar la red neuronal en el Centro de Supercomputación de Castilla y León (SCAYLE), lo que implica el uso de la aplicación PuTTY, el manejo de comandos básicos en Linux y SLURM, y el uso de la aplicación WinSCP.
\end{itemize}