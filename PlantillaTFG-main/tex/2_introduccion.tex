\capitulo{2}{Introducción}

Una de las ventajas que presentan titulaciones como Ingeniería de la Salud es la capacidad para desarrollar proyectos interdisciplinares, es decir, que aúnen técnicas y conocimientos de diversos campos. Ejemplo de ello es el trabajo recogido en esta memoria, ya que en él convergen ideas pertenecientes al ámbito sanitario, biológico y computacional.

Este tipo de iniciativas son capaces de abrir nuevos campos de investigación, ampliar las perspectivas y mejorar la eficiencia en la resolución de problemas. Sin embargo, estas propuestas generalmente son complicadas de entender en su totalidad, debido a la heterogeneidad de conceptos abarcados.

Es por ello que en las siguientes secciones y subsecciones se van a explicar los fundamentos informáticos y fisiológicos esenciales para poder entender el proyecto, independientemente del ámbito al que pertenezca el lector, así como la estructura de la memoria y del resto de materiales entregados.

\section{Estructura de la memoria}

\section{Conceptos teóricos básicos}

Explicación de los conceptos teóricos básicos necesarios para que cualquier miembro del tribunal pueda entender el trabajo realizado.

Esta sección puede contener el número de subsecciones que sean necesarias.\cite{wiki:latex}

\subsection{Retinopatía Diabética}

\subsubsection{Sub Subsección}

\subsection{Redes Neuronales Convolucionales}

En esta sección y el resto de secciones de la memoria puede ser necesario incluir listas de items.

\begin{itemize}
    \item item1
    \item item2
    \item item3
\end{itemize}

Listas enumeradas.

\begin{enumerate}
    \item item1
    \item item2
    \item item3
\end{enumerate}

Figuras, como la figura \ref{fig:escudo} que aparece en la página \pageref{fig:escudo}. 

Puedes aprender más de las figuras en la dirección \url{https://es.overleaf.com/learn/latex/Inserting_Images}

\begin{figure}[h]
    \centering
    \includegraphics[width=0.25\textwidth]{img/escudoSalud.pdf}
    \caption{Pie de la figura}
    \label{fig:escudo}
\end{figure}


También se pueden insertar tablas como \ref{tab:my-table}, que ha sido generada con \url{https://www.tablesgenerator.com/}.

\begin{table}[]
\begin{tabular}{lll}
a & b & c \\
1 & 2 & 3 \\
4 & 5 & 6
\end{tabular}
\caption{}
\label{tab:my-table}
\end{table}

Es necesario que todas las figuras y tablas aparezca referenciadas en el texto, como estos ejemplos.

Todos los conceptos teóricos deben de estar correctamente referenciados en la bibliografía. Por ejemplo, aquí estoy citando la página de \LaTeX{} de Wikipedia \cite{wiki:latex}.

También puede ser necesario utilizar notas al pie \footnote{como por ejemplo esta}, para aclarar algunos conceptos.

\section{Marco de trabajo}

\section{Estado del arte y trabajos relacionados}

En este apartado sería interesante también incluir la justificación del proyecto, qué se pretende hacer en este proyecto que no se haya hecho en otros, qué es lo novedoso que se va a introducir (en este caso adecuar la red para que trabaje sobre imágenes de dispositivos móviles en vez de imágenes de OCT).

Revisión bibliográfica de que se está haciendo en la industria o la academia relativo al problema que se está tratando.

Enumeración y resumen de todos los trabajos relacionados de interés.

