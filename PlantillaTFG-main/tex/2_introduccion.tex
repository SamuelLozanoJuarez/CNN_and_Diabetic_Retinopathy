\capitulo{2}{Introducción} \label{Intro}

\setlength{\parskip}{10pt}

Una de las ventajas que presentan titulaciones como Ingeniería de la Salud es la capacidad para desarrollar proyectos interdisciplinares, es decir, que aúnen técnicas y conocimientos de diversos campos. Ejemplo de ello es el trabajo recogido en esta memoria, ya que en él convergen ideas pertenecientes al ámbito sanitario, biológico y computacional.

Este tipo de iniciativas son capaces de abrir nuevos campos de investigación, ampliar las perspectivas y mejorar la eficiencia en la resolución de problemas. Sin embargo, estas propuestas generalmente son complicadas de entender en su totalidad, debido a la heterogeneidad de conceptos abarcados.

Es por ello que en las siguientes secciones y subsecciones se van a explicar los fundamentos informáticos y fisiológicos esenciales para poder entender el proyecto, independientemente del ámbito al que pertenezca el lector, así como la estructura de la memoria y del resto de materiales entregados.

\titlespacing{\section}{0pt}{0.25cm}{0.15cm}
\section{Estructura de la memoria}

En este documento y sus correspondientes anexos se recoge toda la información del proyecto \textit{Detección del grado de retinopatía diabética mediante redes convolucionales}.

En la memoria principal se incluyen los capítulos detallados en el \hyperref[toc]{Índice}: \hyperref[Obj]{Objetivos}, \hyperref[Intro]{Introducción}, \hyperref[Met]{Metodología}, \hyperref[Conc]{Conclusiones}, \hyperref[Aspe]{Aspectos relevantes del desarrollo del proyecto} y \hyperref[Fut]{Líneas futuras}. Además se incluye la correspondiente bibliografía en la que se citan todas las fuentes empleadas en el proyecto.

Asimismo se entrega una memoria de anexos, que contiene materiales adicionales y ampliaciones de los contenidos de la memoria principal.

Todos los materiales empleados en el proyecto, incluyendo el código \LaTeX de las memorias, los scripts de Python para la creación y entrenamiento de modelos, y las imágenes empleadas en el entrenamiento, pueden encontrarse en el siguiente repositorio de GitHub: \href{https://github.com/SamuelLozanoJuarez/CNN_and_Diabetic_Retinopathy}{CNN and Diabetic Retinopathy}.

\titlespacing{\section}{0pt}{0.25cm}{0.15cm}
\section{Conceptos teóricos básicos}

A continuación se van a explicar los conceptos teóricos fundamentales sobre la retinopatía diabética y las redes neuronales convolucionales, necesarios para entender el contexto en que se desarrolla este proyecto.

\titlespacing{\subsection}{0pt}{0.25cm}{0.15cm}
\subsection{Retinopatía diabética}

La retinopatía diabética (RD) es la complicación más común de la diabetes mellitus (DM), y constituye una de las principales causas de ceguera en edad avanzada \cite{diabetes:JDI, retinopatia:Retinal_and_eye}. Se trata de una microangiopatía diabética, una patología que afecta a los capilares que irrigan la retina, como consecuencia de los altos niveles de glucosa en sangre \cite{retinopatia:chile}.

Esta enfermedad se desarrolla en más de una quinta parte de los individuos diagnosticados con DM a nivel global, siendo esta proporción mayor en la región de América del Norte. En números brutos estamos hablando de más de 100 millones de personas que padecen esta afección en todo el mundo \footnote{Los datos se corresponden con un estudio realizado en el año 2021.} \cite{retinopatia:ophtalmology}.

A pesar de que existen tratamientos que permiten ralentizar, e incluso revertir, la evolución de la enfermedad, estos presentan una mayor eficacia en las primeras etapas de desarrollo de la RD, por lo que un diagnóstico temprano es fundamental a la hora de preservar la visión del paciente \cite{diabetes:JDI}.

\subsubsection{Sub Subsección}

\subsection{Redes Neuronales Convolucionales}

En esta sección y el resto de secciones de la memoria puede ser necesario incluir listas de items.

\begin{itemize}
    \item item1
    \item item2
    \item item3
\end{itemize}

Listas enumeradas.

\begin{enumerate}
    \item item1
    \item item2
    \item item3
\end{enumerate}

Figuras, como la figura \ref{fig:escudo} que aparece en la página \pageref{fig:escudo}. 

Puedes aprender más de las figuras en la dirección \url{https://es.overleaf.com/learn/latex/Inserting_Images}

\begin{figure}[h]
    \centering
    \includegraphics[width=0.25\textwidth]{img/escudoSalud.pdf}
    \caption{Pie de la figura}
    \label{fig:escudo}
\end{figure}


También se pueden insertar tablas como \ref{tab:my-table}, que ha sido generada con \url{https://www.tablesgenerator.com/}.

\begin{table}[]
\begin{tabular}{lll}
a & b & c \\
1 & 2 & 3 \\
4 & 5 & 6
\end{tabular}
\caption{}
\label{tab:my-table}
\end{table}

Es necesario que todas las figuras y tablas aparezca referenciadas en el texto, como estos ejemplos.

Todos los conceptos teóricos deben de estar correctamente referenciados en la bibliografía. Por ejemplo, aquí estoy citando la página de \LaTeX{} de Wikipedia \cite{wiki:latex}.

También puede ser necesario utilizar notas al pie \footnote{como por ejemplo esta}, para aclarar algunos conceptos.

\section{Marco de trabajo}

\section{Estado del arte y trabajos relacionados}

En este apartado sería interesante también incluir la justificación del proyecto, qué se pretende hacer en este proyecto que no se haya hecho en otros, qué es lo novedoso que se va a introducir (en este caso adecuar la red para que trabaje sobre imágenes de dispositivos móviles en vez de imágenes de OCT).

Revisión bibliográfica de que se está haciendo en la industria o la academia relativo al problema que se está tratando.

Enumeración y resumen de todos los trabajos relacionados de interés.

