\apendice{Plan de Proyecto Software}

\section{Introducción}

La planificación de un proyecto es una de las partes más importantes, ya que nos permite distribuir el trabajo a lo largo del tiempo, prevenir la aparición de posibles problemas y solventar los contratiempos que puedan aparecer.

En este Apéndice abordaremos 3 aspectos de la planificación:
\begin{itemize}[itemsep=0.1em]
    \item Planificación temporal
    \item Planificación económica 
    \item Planificación legal o viabilidad legal
\end{itemize}

En el primero de ellos, la planificación temporal, se detallará la metodología empleada y las tareas programadas para cada semana a lo largo del proyecto. Además se evaluará la rigurosidad con la que se han seguido los pasos de la metodología y el grado de cumplimiento de las tareas en cada semana.

En el segundo apartado, la planificación económica, se tratará de estimar el coste total que implicaría desarrollar el proyecto de manera íntegra. Para ello se tendrán en cuenta costes de \textit{hardware}, \textit{software}, coste humano y otros gastos no considerados en los ya mencionados (como el consumo de luz o la conexión a Internet).

Por último se estudiará la viabilidad legal del proyecto, desde el punto de vista clínico (tratamiento de datos de naturaleza sensible y consentimiento informado), y también desde el punto de vista empresarial (registro del producto desarrollado y licencias asociadas al mismo).

\section{Planificación temporal}

Desde el planteamiento inicial del proyecto se apostó por seguir la metodología denominada \textit{CRISP-DM}, siglas que aluden a los términos \textit{CRoss Industry Standard Process for Data Mining} \cite{crispdm:schorer}. 

\textit{CRISP-DM} es una estrategia aplicable a múltiples ámbitos de la industria, pero especialmente empleada en la minería de datos \cite{crispdm:azevedo}, que consiste en 6 fases o etapas iterativas que abarcan desde la comprensión del ámbito en que se va a trabajar (o comprensión del negocio si se traduce directamente del inglés \textit{Business Understanding}), hasta el despliegue de los resultados \cite{crispdm:niaksu,crispdm:schorer}.

Las 6 fases que constituyen el modelo son las siguientes \cite{crispdm:niaksu,crispdm:schorer}:
\begin{enumerate}[itemsep=0.1em]
    \item \textbf{Comprensión del ámbito o negocio (\textit{Business understanding})}. Abarca el establecimiento y comprensión de los objetivos del proyecto y la definición de los criterios de éxito (\textit{p. ej.} las métricas de evaluación que se emplearán). En esta etapa se incluye también la explicación del tipo de minería de datos que se desea realizar y la creación del plan de trabajo.
    \item \textbf{Comprensión de los datos (\textit{Data understanding})}. Incluye la recopilación de datos de distintas fuentes, su exploración y descripción, lo que implica el estudio de los atributos de estos datos. En este paso también se suelen realizar suposiciones sobre qué datos del conjunto total serán útiles para el proyecto.
    \item \textbf{Preparación de los datos (\textit{Data preparation})}. En esta fase se deben determinar los criterios de inclusión y exclusión de los datos, sumado a la posterior limpieza de los datos, eliminando aquellos que no cumplan los requisitos necesarios. También implica la modificación de los datos para mejorar su calidad o eliminar ruido.
    \item \textbf{Modelado (\textit{Modeling})}. Básicamente consiste en seleccionar qué técnica de modelado se desea emplear (en nuestro caso consistiría en configurar la arquitectura de la red neuronal), y explicar por qué se ha decidido seleccionar ese modelo. También se incluye la optimización de los parámetros del modelo.
    \item \textbf{Evaluación (\textit{Evaluation})}. Una vez construido el modelo, este debe ser probado, generando métricas que permitan determinar su rendimiento. Entre las métricas que emplearemos en el proyecto se pueden encontrar \textit{Quadratic Weighted Kappa}, \textit{Balanced Accuracy}, \textit{F-score} y el área bajo la curva ROC. Los resultados de esta evaluación deberán compararse con los objetivos definidos, interpretando los resultados y determinando si el modelo cumple dichos objetivos.
    \item \textbf{Despliegue o implementación (\textit{Deployment})}. El contenido de esta última etapa depende del proyecto. Generalmente suele consistir en la presentación de los resultados de manera organizada mediante un informe, como en nuestro caso. Sin embargo, si a lo largo del proyecto se ha desarrollado un componente software, en esta fase de despliegue también puede llevarse a cabo la distribución del componente en el mercado.
\end{enumerate}

Como se ha mencionado anteriormente, \textit{CRISP-DM} es una metodología iterativa, es decir, estas 6 etapas se van repitiendo en un bucle cuyo orden no está rígidamente definido, sino que la sucesión de las distintas fases estará determinada por el resultado de la etapa finalizada \cite{crispdm:niaksu}, tal y como se muestra en la Figura \ref{crispdm:flujo}.

\begin{figure}[h]
    \centering
    \includegraphics[width=0.65\textwidth]{img/CRISP-DM.png}
    \caption{Flujo de trabajo de la metodología CRISP-DM. Nótese que existe la posibilidad de invertir la dirección del flujo entre algunas etapas. Fuente: \cite{crispdm:flujo}}.
    \label{crispdm:flujo}
\end{figure}

Para llevar a cabo la planificación temporal del proyecto, se emplearon las herramientas de control proporcionadas por GitHub, concretamente la extensión ZenHub. 

Los principios básicos de la planificación temporal fueron los siguientes:

\begin{itemize}[itemsep=0.1em]
    \item Realización de \textit{sprints} con una duración de 2 semanas.
    \item Planificación de las tareas al inicio de cada \textit{sprint}.
    \item Creación y uso de un tablero de trabajo en ZenHub en el que se clasificaban las tareas según su estatus: tareas abiertas sin comenzar (\textit{Backlog}), en progreso, finalizadas y cerradas.
    \item Creación de 7 etiquetas, 6 de ellas correspondientes a las distintas etapas de la metodología de trabajo \textit{CRISP-DM} y una séptima correspondiente a la categoría \textit{update}, tal y como se puede apreciar en la figura \ref{labels:github}. Cada tarea programada fue marcada con una o más etiquetas.
\end{itemize}

\begin{figure}[h]
    \centering
    \includegraphics[width=0.75\textwidth]{img/labels_github.jpg}
    \caption{Etiquetas creadas. En la columna de la izquierda se indica el nombre de la etiqueta, en la columna de la derecha una breve definición de la misma. Fuente: elaboración propia}.
    \label{labels:github}
\end{figure}

A continuación se detallará para cada \textit{sprint} la fecha de inicio y de fin, las tareas inicialmente programadas y una evaluación del cumplimiento de las tareas.

\subsection{Sprint 0}

\section{Planificación económica}

\section{Viabilidad legal}