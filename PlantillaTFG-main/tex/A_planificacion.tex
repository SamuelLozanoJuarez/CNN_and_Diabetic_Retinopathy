\apendice{Plan de Proyecto Software}

\section{Introducción}

La planificación de un proyecto es una de las partes más importantes, ya que nos permite distribuir el trabajo a lo largo del tiempo, prevenir la aparición de posibles problemas y solventar los contratiempos que puedan aparecer.

En este Apéndice abordaremos 3 aspectos de la planificación:
\begin{itemize}[itemsep=0.1em]
    \item Planificación temporal
    \item Planificación económica 
    \item Planificación legal o viabilidad legal
\end{itemize}

En el primero de ellos, la planificación temporal, se detallará la metodología empleada y las tareas programadas para cada semana a lo largo del proyecto. Además se evaluará la rigurosidad con la que se han seguido los pasos de la metodología y el grado de cumplimiento de las tareas programadas en las semanas.

En el segundo apartado, la planificación económica, se tratará de estimar el coste total que implicaría desarrollar el proyecto de manera íntegra. Para ello se tendrán en consideración costes de \textit{hardware}, \textit{software}, coste humano y otros gastos no considerados en los ya mencionados (como el consumo de luz o la conexión a Internet).

Por último se estudiará la viabilidad legal del proyecto, desde el punto de vista clínico (tratamiento de datos de naturaleza sensible y consentimiento informado), y también desde el punto de vista empresarial (registro del producto desarrollado y licencias asociadas al mismo).

\section{Planificación temporal}

\section{Planificación económica}

\section{Viabilidad legal}