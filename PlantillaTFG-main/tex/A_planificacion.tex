\apendice{Plan de Proyecto Software}

\section{Introducción}

La planificación de un proyecto es una de las partes más importantes, ya que nos permite distribuir el trabajo a lo largo del tiempo, prevenir la aparición de posibles problemas y solventar los contratiempos que puedan aparecer.

En este Apéndice abordaremos 3 aspectos de la planificación:
\begin{itemize}[itemsep=0.1em]
    \item Planificación temporal
    \item Planificación económica 
    \item Planificación legal o viabilidad legal
\end{itemize}

En el primero de ellos, la planificación temporal, se detallará la metodología empleada y las tareas programadas para cada semana a lo largo del proyecto. Además se evaluará la rigurosidad con la que se han seguido los pasos de la metodología y el grado de cumplimiento de las tareas en cada semana.

En el segundo apartado, la planificación económica, se tratará de estimar el coste total que implicaría desarrollar el proyecto de manera íntegra. Para ello se tendrán en cuenta costes de \textit{hardware}, \textit{software}, coste humano y otros gastos no considerados en los ya mencionados (como el consumo de luz o la conexión a Internet).

Por último se estudiará la viabilidad legal del proyecto, desde el punto de vista clínico (tratamiento de datos de naturaleza sensible y consentimiento informado), y también desde el punto de vista empresarial (registro del producto desarrollado y licencias asociadas al mismo).

\section{Planificación temporal}

Desde el planteamiento inicial del proyecto se apostó por seguir la metodología denominada \textit{CRISP-DM}, siglas que aluden a los términos \textit{CRoss Industry Standard Process for Data Mining} \cite{crispdm:schorer}. 

\textit{CRISP-DM} es una estrategia aplicable a múltiples ámbitos de la industria, pero especialmente empleada en la minería de datos \cite{crispdm:azevedo}, que consiste en 6 fases o etapas iterativas que abarcan desde la comprensión del ámbito en que se va a trabajar (o comprensión del negocio si se traduce directamente del inglés \textit{Business Understanding}), hasta el despliegue de los resultados \cite{crispdm:schorer,crispdm:niaksu}.

Las 6 fases que constituyen el modelo son las siguientes:
\begin{enumerate}[itemsep=0.1em]
    \item \textbf{Comprensión del ámbito o negocio (\textit{Business understanding})}.
    \item \textbf{Comprensión de los datos (\textit{Data understanding})}
    \item \textbf{Preparación de los datos (\textit{Data preparation})}
    \item \textbf{Modelado (\textit{Modeling})}
    \item \textbf{Evaluación (\textit{Evaluation})}
    \item \textbf{Despliegue (\textit{Deployment})}
\end{enumerate}

Combina ciertos aspectos propios de las metodologías ágiles y otros más característicos de las denominadas estrategias \textit{waterfall}

\section{Planificación económica}

\section{Viabilidad legal}